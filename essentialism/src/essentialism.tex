\documentclass{beamer}
\usepackage{epigraph}
\usetheme{Warsaw}

\renewcommand{\epigraphflush}{flushleft}
\setlength\epigraphwidth{1\textwidth}

\begin{document}

    \title{Essentialism}
    \author{Vasily Ermakov}
    \date{Vilnius, 2023}
    \frame{\titlepage}

    \begin{frame}{Minimalism meaning}
        \begin{block}
            <+->{Origin}
            Originated in visual arts in 20th century.
            Work exposes the essence, essentials or identity of a subject through eliminating all non-essential forms, features or concepts.
        \end{block}
        \begin{block}
            <+->{Modern meanings}
            \begin{itemize}
                \item simple living
                \item decluttering
                \item essentialism
            \end{itemize}
        \end{block}
    \end{frame}

    \begin{frame}{Essentialism meaning}
        \begin{block}
            <+->{Distinction}
            \begin{itemize}
                \item less is more
                \item less but better
            \end{itemize}
        \end{block}
        \begin{block}
            <+->{Context}
            \epigraph{Substantial and rapidly growing numbers of people have choices.}{P. Drucker}
        \end{block}
        \begin{block}
            <+->{Meaning}
            An investment of personal resources in order to operate at the highest point of contribution by doing only what is essential.
        \end{block}
    \end{frame}

    \begin{frame}{Paradigm}
        \begin{block}{}
            What's the difference between paradigm and practice?
        \end{block}
    \end{frame}

    \begin{frame}{What is essential?}
        \begin{block}{}
            An investment of personal resources in order to operate at the highest point of contribution by doing only \textbf{what is essential}.
        \end{block}
    \end{frame}

    \begin{frame}{Personal resources}
        \begin{block}
            <+->{Scarcity problem in economics}
            The gap between limited resources and theoretically limitless wants.
        \end{block}
        \begin{block}
            <+->{}
            An investment of \textbf{personal resources} in order to operate at the highest point of contribution by doing only what is essential.
        \end{block}
        \begin{block}
            <+->{}
            \begin{itemize}
                \item time
                \item energy
                \item attention
                \item willpower
            \end{itemize}
        \end{block}
    \end{frame}

    \begin{frame}{Sources}
        \begin{block}{}
            \begin{itemize}
                \item McKeown, G. (2014). Essentialism. New York: Crown Business.
                \item Covey, S. (2004). The seven habits of highly effective people. New York: Free Press.
                \item Keller, G \& Papasan, J. (2012). The one thing: the surprisingly simple truth behind extraordinary results. Austin: Bard Press.
                \item https://www.investopedia.com/terms/s/scarcity.asp
                \item https://en.wikipedia.org/wiki/Minimalism\_(disambiguation)
                \item https://github.com/Vasily-Ermakov/presentations
            \end{itemize}
        \end{block}
    \end{frame}
\end{document}
